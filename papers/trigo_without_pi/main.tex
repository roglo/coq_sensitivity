\documentclass[11pt]{article}

\usepackage[utf8]{inputenc}
\usepackage[T1]{fontenc}
\usepackage{amsmath,amssymb,amsthm}
\usepackage{hyperref}
\usepackage{graphicx}
\usepackage{geometry}
\geometry{margin=2.5cm}

\title{Trigonometry without $\pi$: a constructive approach}
\author{Daniel de Rauglaudre (alias roglo)}
\date{\today}

\begin{document}

\maketitle

\begin{abstract}
We present a construction of trigonometry where angles are not real
numbers, but pairs $(x,y)$ such that $x^2 + y^2 = 1$. Several classic
trigonometric formulas naturally emerge during this construction,
which leads to defining the division of an angle by an integer using a
convergent sequence. This construction does not require the prior
definition of the constant $\pi$, which is never used. All results are
formally proven using the Coq proof assistant.
\end{abstract}

\section{Introduction}

Pouet.

\section{Conclusion}

Glop.

\paragraph{Code source :}
\url{https://github.com/roglo/coq_sensitivity/tree/master/trigo_without_pi}

\end{document}
