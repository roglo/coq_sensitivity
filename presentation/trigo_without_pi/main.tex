\documentclass{beamer}
\usepackage{amsmath}
\usepackage{pgfpages} % Package to improve the compatibility of watermarks
\usepackage[absolute,overlay]{textpos}
\usepackage{tikz}

\setlength{\TPHorizModule}{1cm} % Définir les unités horizontales
\setlength{\TPVertModule}{1cm}  % Définir les unités verticales
\newcommand*\hstrut[1]{\vrule width0pt height0pt depth#1\relax}

\newcommand{\crossedout}[1]{%
  \tikz[baseline=(X.base)]{
    \node[inner sep=0pt,outer sep=0pt] (X) {#1};
    \draw[red, thick] (X.south west) -- (X.north east);
    \draw[red, thick] (X.north west) -- (X.south east);
  }%
}
\title{Trigonometry Without $\pi$}
\author{Daniel de Rauglaudre}
\date{\today}

% Custom footer to include the slide number in the bottom right in the
% format 1/5, 2/5, ...
\setbeamertemplate{footline}{
  \ifnum\value{framenumber}>2
  \hspace{2em}
  \raisebox{3ex}{trigonometry without $\pi$}
  \fi
  \hfill % Fill space to the left to push the text to the right
  \raisebox{3ex}{\insertframenumber{} / \inserttotalframenumber}
  \hspace{1em} % Shifts the footer text slightly to the right
}
\setbeamertemplate{navigation symbols}{} % Delete navigation icons

\begin{document}

% Title slide
\begin{frame}
    \titlepage
\end{frame}

% Second slide with "DISCLAIMER" in the background, adjusted size and
% bold
\begin{frame}{}
    \begin{tikzpicture}[remember picture, overlay]
        \node[opacity=0.4, rotate=45, scale=4.5] at (current page.center) {
            \textcolor{gray}{\textbf{DISCLAIMER}}
        };
    \end{tikzpicture}
\end{frame}

\begin{frame}{Context}
  \begin{itemize}
    \item close to ``rational trigonometry'':  Norman Wildberger
      \begin{itemize}
      \item indeed, redefinition of ``angles''
      \item but I use ``square roots''
      \item different goals
      \end{itemize}
    \uncover<2->{
    \item all proofs are in Coq
      \begin{itemize}
      \item no use of ``mathematical component''
      \end{itemize}}
  \end{itemize}
\end{frame}

\begin{frame}{Why Trigonometry?}
  \begin{itemize}
  \item not a specific interest of that topic...
  \item wanted to resolve a problem involving trigonometry
  \end{itemize}
\end{frame}

\begin{frame}{n-th root of a Complex}
  \only<1>{\hspace{2.1mm} $x + i y$}
  \only<2>{$\sqrt[n]{x + i y} \hspace{5mm}...$}
  \only<3>{\hspace{2.1mm} $x + i y = \rho \hspace{1mm} e^{i \theta}$}
  \only<4>{$\sqrt[n]{x + i y} = \sqrt[n]\rho \hspace{1mm}
    e^{\frac{i\theta}{n}}$}
  \only<5>{\hspace{2.1mm} $x + i y = \rho \hspace{1mm} e^{i \theta}$}
  \only<6->{\hspace{2.1mm} $x + i y = \rho \hspace{1mm}
    (cos \; \theta + i \; sin \; \theta)$}
\end{frame}

\begin{frame}{What is an Angle?}
  \begin{textblock}{7}(1,3)
    % "1,3" représente (x, y) en cm, et "7" est la largeur du bloc texte
    \only<1>{
      $Angle ::= \mathbb{R}$ \\
    }
    \only<2>{
      \crossedout{$Angle ::= \mathbb{R}$} \\
      \vspace{5mm}
      $Angle ::= \{ \; (x, y) \; | \; x^2+y^2=1 \; \}$
    }
    \only<3>{
      \crossedout{$Angle ::= \mathbb{R}$} \\
      \vspace{5mm}
      $Angle ::= \{ \; (x, y) \; | \; x^2+y^2=1 \; \}$ \\
      \vspace{5mm}
      $\theta = (x, \; y, \; x^2 \mathord{+} y^2 \mathord{=} 1 )$
    }
    \only<4->{
      \crossedout{$Angle ::= \mathbb{R}$} \\
      \vspace{5mm}
      $Angle ::= \{ \; (x, y) \; | \; x^2+y^2=1 \; \}$ \\
      \vspace{5mm}
      $\theta = (x, \; y, \; x^2 \mathord{+} y^2 \mathord{=} 1 )$ \\
      \vspace{2mm}
      \begin{itemize}
      \item $cos \; \theta = x$
      \item $sin \; \theta = y$
      \only<5->{\item $cos^2 \; \theta + sin^2 \; \theta = 1$}
      \end{itemize}
    }
  \end{textblock}
\end{frame}

\begin{frame}{Sum of Angles}
  \begin{textblock}{7}(1,3)
    $
    \underbrace{\hstrut{2mm} (x, y)}_{\text{angle}} +
    \underbrace{\hstrut{2mm} (x', y')}_{\text{angle}} ::=
    \underbrace{\hstrut{2mm} (x'', y'')}_{\text{angle}}
    $
    \only<2->{
      \vspace{5mm}
      \begin{itemize}
      \item $cos (a \mathord{+} b) = cos \; a \; cos \; b - sin \; a
        \; sin \; b$
      \item $sin (a \mathord{+} b) = sin \; a \; cos \; b + cos \; a
        \; sin \; b$
      \end{itemize}
    }
    \only<3->{
      \vspace{4mm}
      $(x, y) + (x', y') ::= (x x' - y y', x y' + x' y)$
    }
  \end{textblock}
\end{frame}

\begin{frame}{Additive Group}
  \begin{textblock}{7}(1,2)
    \begin{itemize}
    \item $(x, y) + (x', y') ::= (x x' - y y', x y' + x' y)$
    \item $0_{angle} ::= (1, 0)$
    \item $+$ is commutative
    \item $+$ is associative
    \only<2->{\item $- (x, y) ::= (x, - y)$}
    \end{itemize}
    \only<3->{
      \vspace{4mm}
      multiplication by a natural
      \begin{itemize}
      \item
        $n \; \theta ::=
        \underbrace{\theta + \theta + ... + \theta}_{n \; times}$
      \end{itemize}
    }
  \end{textblock}
\end{frame}

\begin{frame}{Division by a natural}
  \begin{textblock}{7}(1,2)
    De Moivre's formula \\
    \hspace{5mm} $(cos \; \theta + i \; sin \; \theta)^n = cos (n
    \theta) + i \; sin (n \theta)$ \\
    \ \\
    \only<2->{
      Trying to define a division by a natural \\
      \hspace{5mm} $(cos (\frac{\theta}{n}) + i \; sin
      (\frac{\theta}{n}))^n = cos \; \theta + i \; sin \; \theta$ \\
      \ \\
    }
    \only<3->{
      Equation to resolve \\
      \hspace{5mm} $(x + i \; y)^n = u + i \; v$
    }
  \end{textblock}
\end{frame}

\begin{frame}{Division by 2}
  $\displaystyle \frac{(x, \; y)}{2} ::=
  \left( \epsilon (y) \sqrt{\frac{1 + x}{2}}, \; \sqrt{\frac{1 -
      x}{2}} \; \right)$
  \\
  \vspace{1cm}
  ... but how to divide by any other natural number?
\end{frame}

\begin{frame}{Division by a natural}
  \begin{textblock}{10}(1,2)
    computing $1$ divided by $n$ in binary \\
    \hspace{1cm}
    $1 / n = 0.a_1a_2a_3...$
    \hspace{2cm}$a_i = 0 \; or \; 1$
    \only<2->{
      \\
      \vspace{3mm}
      for example \\
      \hspace{1cm}
      $1 / 2 = 0.10000000...$ \\
      \hspace{1cm}
      $1 / 3 = 0.01010101...$ \\
      \hspace{1cm}
      $1 / 4 = 0.00100000...$ \\
      \hspace{1cm}
      $1 / 5 = 0.00110011...$ \\
      \vspace{3mm}
      \only<2>{and so on...}
      \only<3->{we have \\
        \hspace{1cm}
        $\displaystyle 0.a_1a_2a_3... = \sum_{i=1}^\infty{\; \frac{a_i}{2^i}}$
      }
    }
  \end{textblock}
\end{frame}

\begin{frame}{Trying to define the division of an angle by a natural}
  \begin{textblock}{10}(1,2)
    $\displaystyle \frac{\theta}{n} ::= \sum_{i=1}^\infty{\; \frac{a_i
        \; \theta}{2^i}}$
  \end{textblock}
\end{frame}

% Une diapositive de conclusion
\begin{frame}{Conclusion}
    \begin{itemize}
        \item Summary of the results.
        \item Implications of this new approach.
        \item Future directions for research.
    \end{itemize}
\end{frame}

\end{document}
