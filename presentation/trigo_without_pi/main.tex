\documentclass{beamer}
\usepackage{tikz}
\usepackage{pgfpages} % Package to improve the compatibility of watermarks

\title{Trigonometry Without $\pi$}
\author{Daniel de Rauglaudre}
\date{\today}

% Custom footer to include the slide number in the bottom right in the
% format 1/5, 2/5, ...
\setbeamertemplate{footline}{
   \hfill % Fill space to the left to push the text to the right
   \insertframenumber{} / \inserttotalframenumber
   \hspace{1em} % Shifts the footer text slightly to the right
}
\setbeamertemplate{navigation symbols}{} % Delete navigation icons

\begin{document}

% Title slide
\begin{frame}
    \titlepage
\end{frame}

% Second slide with "DISCLAIMER" in the background, adjusted size and
% bold
\begin{frame}{}
    \begin{tikzpicture}[remember picture, overlay]
        \node[opacity=0.4, rotate=45, scale=4.5] at (current page.center) {
            \textcolor{gray}{\textbf{DISCLAIMER}}
        };
    \end{tikzpicture}
\end{frame}

% Première diapositive d'introduction
\begin{frame}{Introduction}
    \begin{itemize}
        \item What is $\pi$?
        \item Why do we redefine trigonometry without $\pi$?
        \item Overview of the new approach.
    \end{itemize}
\end{frame}

% Une diapositive avec des mathématiques
\begin{frame}{Angles as Triplets}
    Angles are defined as triplets $(x, y, x^2 + y^2 = 1)$.
    \begin{equation}
        \text{Addition of angles: } (x_1, y_1) + (x_2, y_2) = (\text{expression involving } x_1, y_1, x_2, y_2)
    \end{equation}
\end{frame}

% Une diapositive de conclusion
\begin{frame}{Conclusion}
    \begin{itemize}
        \item Summary of the results.
        \item Implications of this new approach.
        \item Future directions for research.
    \end{itemize}
\end{frame}

\end{document}
