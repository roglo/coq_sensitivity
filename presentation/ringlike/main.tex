% $Id: principal.tex,v 1.39 2012/02/02 19:21:01 deraugla Exp $

\documentclass[t]{beamer}
\usepackage[utf8]{inputenc}
\usepackage[french]{babel}
\usepackage{xcolor}
\usepackage[normalem]{ulem}
\usepackage{wasysym}
\usepackage{url}
\usepackage{listings}
\usepackage{verbatim}
\usepackage{moreverb}
\usepackage{gastex}
\usepackage{graphicx} 
\usepackage{multicol}

\mode<presentation> {
  \usetheme{Rocquencourt}
  \setbeamercovered{highly dynamic}
%  \setbeamersize{text margin left=0.5cm}
}

\AtBeginSection[]  {
\begin{frame}<beamer>
 \frametitle{Plan}
 \setcounter{tocdepth}{2}
 \tableofcontents[currentsection,sectionstyle=show/shaded,
                  subsectionstyle=show/show/hide]
\end{frame}
}


%========================================================================%
% Blocks
%========================================================================%

\definecolor{myblue}{rgb}{0, 0.08, 0.55}

\setbeamertemplate{blocks}[rounded][shadow=true]
\setbeamercolor{block title}{bg=myblue, fg=white}    
\setbeamercolor{block body}{bg=white, fg=black}     

\setbeamercolor{block title example}{bg=green,fg=white}   
\setbeamercolor{block body example}{bg=white,fg=black}   

\setbeamercolor{block title alerted}{bg=red,fg=white}   
\setbeamercolor{block body alerted}{bg=white,fg=black}   



%========================================================================%
% New command 
%========================================================================%
\newcommand{\frm}[2]{\frame{\frametitle{#1}\
#2}}

\newcommand{\parag}[1]{{\bf\textcolor{inriablue}{#1}}\\}
\newcommand{\syndex}{SynDEx}
\newcommand{\oc}{OCaml}
\newcommand{\vsp}{\vspace{0.4cm}}
\newcommand{\vspd}{\vspace{0.2cm}}
\newcommand{\hsp}{\hspace{0.3cm}}

			


% Images
\pgfdeclareimage[interpolate=true,height=0.7cm]{ordo1}{images/ordo1}
\pgfdeclareimage[interpolate=true,height=0.7cm]{ordo1-0}{images/ordo1-0}
\pgfdeclareimage[interpolate=true,height=2.0cm]{ordo2}{images/ordo2}
\pgfdeclareimage[interpolate=true,height=2.0cm]{ordo2ech}{images/ordo2echec}
\pgfdeclareimage[interpolate=true,height=2.0cm]{ordo2ech2}{images/ordo2echec2}
\pgfdeclareimage[interpolate=true,width=8.0cm]{ordo2seul}{images/ordo2seul}
\pgfdeclareimage[interpolate=true,width=8.0cm]{ordo2pgcd}{images/ordo2pgcd}
\pgfdeclareimage[interpolate=true,height=3.0cm]{ordo3pgcd}{images/ordo3pgcd}
\pgfdeclareimage[interpolate=true,height=3.0cm]{ordo3}{images/ordo3}
\pgfdeclareimage[interpolate=true,height=3.0cm]{ordo3fond}{images/ordo3fond}
\pgfdeclareimage[interpolate=true,height=1.5cm]{ordo12}{images/ordo12}
\pgfdeclareimage[interpolate=true,height=1.451cm]{ordo13}{images/ordo13}
\pgfdeclareimage[interpolate=true,height=1.44cm]{ordo23}{images/ordo23}
\pgfdeclareimage[interpolate=true,height=2cm]{ordo3coro}{images/ordo3coro}

\begin{document}

\title[Irif - Inria - 24th dec 2020] 
{Ring-like algebraic structures \\
  formalized in Coq}
\author{Daniel de Rauglaudre}
\institute{\small Irif - Inria}
\date{24th december 2020}
\frame{\titlepage}

% - work made in Coq
% - Ring-like algebraic structures formalizes in Coq

\frm{\only<1-4>{Reminder}\only<5>{Ring-likes}}{

  \begin{tabular}{lll}
    Semiring \only<1>{?} &
    \only<2->
        {: & $\mathbb{N}$} \\
    Ring \only<1-2>{?} &
    \only<3->{: & $\mathbb{Z}$, matrices} \\
    Field \only<1-3>{?} &
    \only<4->{: & $\mathbb{Q}$, $\mathbb{R}$, $\mathbb{C}$, ...} \\
  \end{tabular}

  \only<5> {
    \vsp All of them are ``ring-like'':
    \begin{itemize}
    \item Two operations: $+$ and $*$
    \item A lot of axioms in common
  \end{itemize}}
}

\frm{Record}{

  \begin{tabular}{ll}
  has\_opposite : bool & \only<9->{is\_archimedean : bool} \\
  has\_inverse : bool & \only<9->{is\_complete : bool} \\
  \only<2->{is\_commutative : bool} &
    \only<9->{is\_algebraically\_closed : bool} \\
  \only<3->{is\_ordered : bool} & \only<9->{is\_finite : bool} \\
  \only<4->{is\_integral : bool} & \only<9->{...} \\
  \only<5->{characteristic : nat} &
  \only<10->{consistent :} \\
  \only<6->{has\_decidable\_equality : bool} &
  \only<10->{\hsp not} \\
  \only<7->{has\_1\_different\_from\_0 : bool} &
  \only<10->{\hsp \hsp (is\_algebraically\_closed and} \\
  \only<8->{is\_euclidean : bool} &
  \only<10->{\hsp \hsp is\_finite)} \\
  \only<8->{is\_factorial : bool \\
    is\_principal : bool }
  \end{tabular}

}

%%%%%

\section[Plan]{}
\frm{Plan}{
\setcounter{tocdepth}{1}
\tableofcontents
}

\section{introduction}

\frm{Introduction} {

  Motivation :
  \begin{itemize}
  \item Systèmes embarqués : sécurité.
  \item Tester preuves formelles.
  \item Projet P (Oséo).
  \end{itemize}
  \vsp
  Logiciels d'ordonnancement :
  \begin{itemize}
  \item SynDEx de l'équipe Aoste (multiprocesseur).
  \item Dameid.
  \end{itemize}
}

\frm{Exemple de trois tâches}{

  tâche $\tau_1$ : \{phase = 0; durée = 1; période = 6\} \\
  tâche $\tau_2$ : \{phase = 3; durée = 1; période = 10\} \\
  tâche $\tau_3$ : \{phase = 1; durée = 2; période = 15\}

  \begin{center}
    \pgfuseimage{ordo3}
  \end{center}
}

\section{théorème}

\frm{Théorème (Jan Korst - 1992)}{
  
  Une liste de tâches \{$\tau_1$ .. $\tau_n$\} est ordonnançable par l'algorithme de \\
  {\em périodicité stricte non préemptive}, si et seulement si :\\
  \begin{center}
    $\forall$ $\tau_i$ = ($\phi_i$, $d_i$, $P_i$) $\in$ \{$\tau_1$ .. $\tau_n$\} \\
    $\forall$ $\tau_j$ = ($\phi_j$, $d_j$, $P_j$) $\in$ \{$\tau_1$ .. $\tau_n$\} \\
    avec i $\ne$ j et $\phi_i$ $\le$ $\phi_j$ \\
    \vsp
    on ait :\\
    \vsp
    \textcolor{blue}{$d_i$ $\le$ ($\phi_j$ - $\phi_i$)\ mod\ $G_{i,j}$ $\le$ $G_{i,j}$ - $d_j$} \\
    \vsp
    où $G_{i,j}$ = pgcd ($P_i$, $P_j$) \\
  \end{center}
}

% - ne dit pas s'il y a une solution à ce problème
% - quelle est l'idée de cette formule magique ?
% - juste une intuition...

\frm{Représentation de $d_i$ + $d_j$ $\le$ $G_{i,j}$}{

  \begin{center}
    \hspace{0.2cm} \pgfuseimage{ordo2seul} \\
    le pgcd des périodes vaut 4 :\\
    \vsp
    \hspace{0cm} \pgfuseimage{ordo2pgcd} \\
  \end{center}
}

\section{preuve formelle}

\subsection{modélisation}

\frm{Structures de données + Exemple}{

  Tâche : \\
  \hspace{2cm} Record \textcolor{blue}{Task} := \\
  \hspace{2.5cm}  \{ \textcolor{blue}{id} : nat ; \textcolor{blue}{phase} : nat ; \textcolor{blue}{duration} : nat ; \textcolor{blue}{period} : nat \}.\\
  \vsp
  Unité de temps : \\
  \vsp
  \hspace{2cm} Inductive \textcolor{blue}{time\_unit} := \\
  \hspace{2.2cm}  $|$ \textcolor{blue}{E} : time\_unit \hspace{2cm} (* processeur libre *)\\
  \hspace{2.2cm}  $|$ \textcolor{blue}{A} : nat $\rightarrow$ time\_unit. \hspace{0.7cm} (* processeur occupé *) \\
  \vsp
  Exemple : \\
  \vsp
  \hspace{2cm} A 1, A 3, A 3, A 2, E, E, A 1, E, E, E, E, E, A 1, ...
}

% - en bleu, ce qui est défini
% - en noir, éléments de syntaxe
% - transition : qu'est-ce qui va construire ces unités de temps ?

\frm{Fonctions}{

  \textcolor{blue}{nth\_time\_unit} : nat $\rightarrow$ Task $\rightarrow$ time\_unit \\
  \begin{itemize}
  \item renvoit la $t^{ieme}$ unité de temps pour la tâche $\tau$
  \end{itemize}
  \vsp
  \textcolor{blue}{merge\_unit\_list} : nat $\rightarrow$ list Task $\rightarrow$
  option time\_unit \\
  \begin{itemize}
  \item renvoit la $t^{ieme}$ unité de temps pour la liste de tâches $\tau$l
  \end{itemize}
  \vsp
  \textcolor{blue}{schedulable} : list Task $\rightarrow$ Prop := \\
  \begin{itemize}
  \item $\forall t$, merge\_unit\_list t $\tau$l $\ne$ None
  \end{itemize}

}

% - par exemple durée = 3, période 10 : A A A E E E E E E E E
% - merge_unit_list peut échouer si conflit entre tâches
% - nth_time_unit : 5 lignes de code
% - merge_unit_list : 14 lignes de code
% - code de ces fonctions dans l'article

\subsection{théorème}

\frm{Condition nécessaire}{

  Si, pour tout couple de tâches ($\tau_i$, $\tau_j$) de $\tau$l, \\
  \hsp\hsp $d_i$ $\le$ ($\phi_j$ - $\phi_i$)\ mod\ $G_{i,j}$ $\le$ $G_{i,j}$ - $d_j$ \\
  alors \\
  \hsp\hsp $\forall$ t, merge\_unit\_list t $\tau$l $\ne$ None.

  \vsp

  \textcolor{blue}{Preuve} : \\
  \hsp Par contradiction. On suppose donc que : \\
  \hsp\hsp\hsp $\exists$ $\tau_i$ $\tau_j$ t / nth\_time\_unit t $\tau_i$ = A i \\
  \hsp\hspace{2.25cm} nth\_time\_unit t $\tau_j$ = A j \\

  \vsp
  \hsp Puis, considérations sur les opérations élémentaires, \\
  \hsp les inégalités et les modulos. Voir article.

}

\frm{Réciproque}{

  On doit montrer que si \\
  \hsp\hsp $\forall$ t, merge\_unit\_list t $\tau$l $\ne$ None, \\

  alors, pour tout couple de tâches ($\tau_i$, $\tau_j$) de $\tau$l, \\
  \hsp\hsp $d_i$ $\le$ ($\phi_j$ - $\phi_i$)\ mod\ $G_{i,j}$ $\le$ $G_{i,j}$ - $d_j$ \\
  \vsp
  \textcolor{blue}{Preuve} : par la contraposée. \\
  \vsp
   mais... mais... est-ce que ça va résoudre le problème ?
}

% - la contraposée n'implique pas forcément l'implication !

\frm{Contraposée... problème !}{

  En logique intuitioniste : \\
  \hsp\hsp (A $\implies$ B) $\implies$ ($\bar{B}$ $\implies$ $\bar{A}$) \\
  \vsp
  Mais la réciproque n'est pas vraie, en général. \\
  \vsp
  Sauf si B est \textcolor{red}{décidable} ! \\
  \vsp
  B = \\
  \hsp\hsp Pour tout couple de tâches ($\tau_i$, $\tau_j$) de $\tau$l, \\
  \hsp\hsp\hsp\hsp $d_i$ $\le$ ($\phi_j$ - $\phi_i$)\ mod\ $G_{i,j}$ $\le$ $G_{i,j}$ - $d_j$ \\
}

\frm{Contraposée de la réciproque} {

  S'il existe un couple de tâches ($\tau_i$, $\tau_j$) de $\tau$l tel que : \\
  \hsp\hsp $d_i$ $>$ ($\phi_j$ - $\phi_i$)\ mod\ $G_{i,j}$ \textcolor{green}{$\vee$} \\
  \hsp\hsp ($\phi_j$ - $\phi_i$)\ mod\ $G_{i,j}$ $>$ $G_{i,j}$ - $d_j$ \\
  alors \\
  \hsp\hsp $\exists$ t, merge\_unit\_list t $\tau$l = None. \\
  \vsp
  \textcolor{blue}{Preuve} : théorème de Bachet-Bézout. \\
  \vsp
  \only<2->{
  \hsp\hsp $\forall$ a b $\in$ $\mathbb{N}$, $\exists$ u v $\in$ $\mathbb{Z}$ / a.u + b.v = pgcd(a, b) \\
  \vsp}
  \only<3->{
  Bachet-Bézout sur les entiers naturels : \\
  \vsp
  \hsp\hsp $\forall$ a b $\in$ $\mathbb{N}$, $\exists$ u v $\in$ \textcolor{blue}{$\mathbb{N}$} / a.u = b.v + pgcd(a, b)}
}

\subsection{corollaires}

\frm{Corollaire 1 : somme des durées} {

  Une liste de tâches \{$\tau_1$ .. $\tau_n$\} est ordonnançable si :
  \begin{center}
    $\sum_{\tau_k}$ $d_k$ $\le$ pgcd $P_k$
  \end{center}
  On choisit les phases : $\phi_1$ = 0 et $\phi_{i+1}$ = $\phi_i$ + $d_i$ \\
  \vsp
  Réciproque fausse en toute généralité. Exemple :
  \begin{center}
    \{$d_1$=1; $P_1$=4 \}
    \{$d_2$=1; $P_2$=6 \}
    \{$d_3$=1; $P_3$=4 \} \\
    \vsp
    \pgfuseimage{ordo3coro}
  \end{center}
}

\frm{Corollaire 2 : deux tâches} {

  Deux tâches $\tau_1$ et $\tau_2$ sont ordonnançables si et seulement si :
  \begin{center}
    $d_1$ + $d_2$ $\le$ pgcd($P_1$, $P_2$)
  \end{center}
  Phases : $\phi_1$ = 0 et $\phi_2$ = $d_1$ \\
  \vsp
  Réciproque vraie dans ce cas.

}

\section{conclusion}

\frm{Concrètement...} {

  \begin{itemize}
  \item début 2011
  \item 2 mois de développement (débutant Coq)
  \item 2000 lignes de Coq
  \item sources sur le Web :
    \url{http://pauillac.inria.fr/~ddr/publi/coq_korst/}
  \item travail clos
  \end{itemize}
  
}

\frm{Conclusion et Travaux futurs} {

  \begin{itemize}
  \item Preuves {\em formelles} possibles dans ce domaine.
  \item Opinion perso : Coq = révolution dans les maths !
  \item Autre théorème en cours : tâches préemptives avec priorité.
  \item Preuves formelles de {\em programmes} embarqués ? Projet P (Oséo).
  \item Preuve formelle du noyau de SynDEx ?
  \item Tactiques : {\em ssreflect} de Microsoft Research-Inria ?
  \end{itemize}

}

\end{document}
